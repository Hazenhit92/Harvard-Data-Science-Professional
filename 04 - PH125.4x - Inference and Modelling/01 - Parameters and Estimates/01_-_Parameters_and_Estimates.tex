\documentclass[]{article}
\usepackage{lmodern}
\usepackage{amssymb,amsmath}
\usepackage{ifxetex,ifluatex}
\usepackage{fixltx2e} % provides \textsubscript
\ifnum 0\ifxetex 1\fi\ifluatex 1\fi=0 % if pdftex
  \usepackage[T1]{fontenc}
  \usepackage[utf8]{inputenc}
\else % if luatex or xelatex
  \ifxetex
    \usepackage{mathspec}
  \else
    \usepackage{fontspec}
  \fi
  \defaultfontfeatures{Ligatures=TeX,Scale=MatchLowercase}
\fi
% use upquote if available, for straight quotes in verbatim environments
\IfFileExists{upquote.sty}{\usepackage{upquote}}{}
% use microtype if available
\IfFileExists{microtype.sty}{%
\usepackage{microtype}
\UseMicrotypeSet[protrusion]{basicmath} % disable protrusion for tt fonts
}{}
\usepackage[margin=1in]{geometry}
\usepackage{hyperref}
\hypersetup{unicode=true,
            pdftitle={Assessment 01 - Parameters and Estimates},
            pdfauthor={Gabriele Mineo - Harvard Data Science Professional},
            pdfborder={0 0 0},
            breaklinks=true}
\urlstyle{same}  % don't use monospace font for urls
\usepackage{color}
\usepackage{fancyvrb}
\newcommand{\VerbBar}{|}
\newcommand{\VERB}{\Verb[commandchars=\\\{\}]}
\DefineVerbatimEnvironment{Highlighting}{Verbatim}{commandchars=\\\{\}}
% Add ',fontsize=\small' for more characters per line
\usepackage{framed}
\definecolor{shadecolor}{RGB}{248,248,248}
\newenvironment{Shaded}{\begin{snugshade}}{\end{snugshade}}
\newcommand{\KeywordTok}[1]{\textcolor[rgb]{0.13,0.29,0.53}{\textbf{#1}}}
\newcommand{\DataTypeTok}[1]{\textcolor[rgb]{0.13,0.29,0.53}{#1}}
\newcommand{\DecValTok}[1]{\textcolor[rgb]{0.00,0.00,0.81}{#1}}
\newcommand{\BaseNTok}[1]{\textcolor[rgb]{0.00,0.00,0.81}{#1}}
\newcommand{\FloatTok}[1]{\textcolor[rgb]{0.00,0.00,0.81}{#1}}
\newcommand{\ConstantTok}[1]{\textcolor[rgb]{0.00,0.00,0.00}{#1}}
\newcommand{\CharTok}[1]{\textcolor[rgb]{0.31,0.60,0.02}{#1}}
\newcommand{\SpecialCharTok}[1]{\textcolor[rgb]{0.00,0.00,0.00}{#1}}
\newcommand{\StringTok}[1]{\textcolor[rgb]{0.31,0.60,0.02}{#1}}
\newcommand{\VerbatimStringTok}[1]{\textcolor[rgb]{0.31,0.60,0.02}{#1}}
\newcommand{\SpecialStringTok}[1]{\textcolor[rgb]{0.31,0.60,0.02}{#1}}
\newcommand{\ImportTok}[1]{#1}
\newcommand{\CommentTok}[1]{\textcolor[rgb]{0.56,0.35,0.01}{\textit{#1}}}
\newcommand{\DocumentationTok}[1]{\textcolor[rgb]{0.56,0.35,0.01}{\textbf{\textit{#1}}}}
\newcommand{\AnnotationTok}[1]{\textcolor[rgb]{0.56,0.35,0.01}{\textbf{\textit{#1}}}}
\newcommand{\CommentVarTok}[1]{\textcolor[rgb]{0.56,0.35,0.01}{\textbf{\textit{#1}}}}
\newcommand{\OtherTok}[1]{\textcolor[rgb]{0.56,0.35,0.01}{#1}}
\newcommand{\FunctionTok}[1]{\textcolor[rgb]{0.00,0.00,0.00}{#1}}
\newcommand{\VariableTok}[1]{\textcolor[rgb]{0.00,0.00,0.00}{#1}}
\newcommand{\ControlFlowTok}[1]{\textcolor[rgb]{0.13,0.29,0.53}{\textbf{#1}}}
\newcommand{\OperatorTok}[1]{\textcolor[rgb]{0.81,0.36,0.00}{\textbf{#1}}}
\newcommand{\BuiltInTok}[1]{#1}
\newcommand{\ExtensionTok}[1]{#1}
\newcommand{\PreprocessorTok}[1]{\textcolor[rgb]{0.56,0.35,0.01}{\textit{#1}}}
\newcommand{\AttributeTok}[1]{\textcolor[rgb]{0.77,0.63,0.00}{#1}}
\newcommand{\RegionMarkerTok}[1]{#1}
\newcommand{\InformationTok}[1]{\textcolor[rgb]{0.56,0.35,0.01}{\textbf{\textit{#1}}}}
\newcommand{\WarningTok}[1]{\textcolor[rgb]{0.56,0.35,0.01}{\textbf{\textit{#1}}}}
\newcommand{\AlertTok}[1]{\textcolor[rgb]{0.94,0.16,0.16}{#1}}
\newcommand{\ErrorTok}[1]{\textcolor[rgb]{0.64,0.00,0.00}{\textbf{#1}}}
\newcommand{\NormalTok}[1]{#1}
\usepackage{graphicx,grffile}
\makeatletter
\def\maxwidth{\ifdim\Gin@nat@width>\linewidth\linewidth\else\Gin@nat@width\fi}
\def\maxheight{\ifdim\Gin@nat@height>\textheight\textheight\else\Gin@nat@height\fi}
\makeatother
% Scale images if necessary, so that they will not overflow the page
% margins by default, and it is still possible to overwrite the defaults
% using explicit options in \includegraphics[width, height, ...]{}
\setkeys{Gin}{width=\maxwidth,height=\maxheight,keepaspectratio}
\IfFileExists{parskip.sty}{%
\usepackage{parskip}
}{% else
\setlength{\parindent}{0pt}
\setlength{\parskip}{6pt plus 2pt minus 1pt}
}
\setlength{\emergencystretch}{3em}  % prevent overfull lines
\providecommand{\tightlist}{%
  \setlength{\itemsep}{0pt}\setlength{\parskip}{0pt}}
\setcounter{secnumdepth}{0}
% Redefines (sub)paragraphs to behave more like sections
\ifx\paragraph\undefined\else
\let\oldparagraph\paragraph
\renewcommand{\paragraph}[1]{\oldparagraph{#1}\mbox{}}
\fi
\ifx\subparagraph\undefined\else
\let\oldsubparagraph\subparagraph
\renewcommand{\subparagraph}[1]{\oldsubparagraph{#1}\mbox{}}
\fi

%%% Use protect on footnotes to avoid problems with footnotes in titles
\let\rmarkdownfootnote\footnote%
\def\footnote{\protect\rmarkdownfootnote}

%%% Change title format to be more compact
\usepackage{titling}

% Create subtitle command for use in maketitle
\newcommand{\subtitle}[1]{
  \posttitle{
    \begin{center}\large#1\end{center}
    }
}

\setlength{\droptitle}{-2em}

  \title{Assessment 01 - Parameters and Estimates}
    \pretitle{\vspace{\droptitle}\centering\huge}
  \posttitle{\par}
    \author{Gabriele Mineo - Harvard Data Science Professional}
    \preauthor{\centering\large\emph}
  \postauthor{\par}
    \date{}
    \predate{}\postdate{}
  

\begin{document}
\maketitle

\subsection{\texorpdfstring{\textbf{Polling - expected value of
S}}{Polling - expected value of S}}\label{polling---expected-value-of-s}

Suppose you poll a population in which a proportion \(\ p\) of voters
are Democrats and \(\ 1−p\) are Republicans. Your sample size is
\(\ N=25\). Consider the random variable \(\ S\), which is the total
number of Democrats in your sample.

What is the expected value of this random variable \(\ S\)?

Possible Answers

\begin{itemize}
\tightlist
\item
  \(\ E(S)=25(1−p)\)
\item
  \(\ E(S)=25p\) {[}X{]}
\item
  \(\ E(S)=\sqrt{25 p (1-p)}\)
\item
  \(\ E(S)=p\)
\end{itemize}

\subsection{\texorpdfstring{\textbf{Polling - standard error of
S}}{Polling - standard error of S}}\label{polling---standard-error-of-s}

Again, consider the random variable S, which is the total number of
Democrats in your sample of 25 voters. The variable p describes the
proportion of Democrats in the sample, whereas 1−p describes the
proportion of Republicans.

What is the standard error of S?

Possible Answers

\begin{itemize}
\tightlist
\item
  \(\ SE(S)=25p(1−p)\)
\item
  \(\ SE(S)=\sqrt{25p}\)
\item
  \(\ SE(S)=25(1−p)\)
\item
  \(\ SE(S)=\sqrt{25 p (1-p)}\) {[}X{]}
\end{itemize}

\subsection{\texorpdfstring{\textbf{Polling - expected value of
X-bar}}{Polling - expected value of X-bar}}\label{polling---expected-value-of-x-bar}

Consider the random variable \(\ S/N\), which is equivalent to the
sample average that we have been denoting as \(\ \bar{X}\). The variable
\(\ N\) represents the sample size and \(\ p\) is the proportion of
Democrats in the population.

What is the expected value of \(\ \bar{X}\)?

Possible Answers

\begin{itemize}
\tightlist
\item
  \(\ E(\bar{X})=p\) {[}X{]}
\item
  \(\ E(\bar{X})=Np\)
\item
  \(\ E(\bar{X})=N(1−p)\)
\item
  \(\ E(\bar{X})=1−p\)
\end{itemize}

\subsection{\texorpdfstring{\textbf{Polling - standard error of
X-bar}}{Polling - standard error of X-bar}}\label{polling---standard-error-of-x-bar}

What is the standard error of the sample average, \(\ \bar{X}\)?

The variable \(\ N\) represents the sample size and \(\ p\) is the
proportion of Democrats in the population.

Possible Answers

\begin{itemize}
\tightlist
\item
  \(\ SE(\bar{X})=\sqrt{Np(1−p)}\)
\item
  \(\ SE(\bar{X})=\sqrt{p(1−p)/N}\) {[}X{]}
\item
  \(\ SE(\bar{X})=\sqrt{p(1−p)}\)
\item
  \(\ SE(\bar{X})=\sqrt{N}\)
\end{itemize}

\subsection{\texorpdfstring{\textbf{se versus
p}}{se versus p}}\label{se-versus-p}

Write a line of code that calculates the standard error \texttt{se} of a
sample average when you poll 25 people in the population. Generate a
sequence of 100 proportions of Democrats \texttt{p} that vary from 0 (no
Democrats) to 1 (all Democrats).

Plot \texttt{se} versus \texttt{p} for the 100 different proportions.

Instructions

\begin{itemize}
\tightlist
\item
  Use the \texttt{seq} function to generate a vector of 100 values of
  \texttt{p} that range from 0 to 1.
\item
  Use the \texttt{sqrt} function to generate a vector of standard errors
  for all values of \texttt{p}.
\item
  Use the \texttt{plot} function to generate a plot with \texttt{p} on
  the x-axis and \texttt{se} on the y-axis.
\end{itemize}

\begin{Shaded}
\begin{Highlighting}[]
\CommentTok{# `N` represents the number of people polled}
\NormalTok{N <-}\StringTok{ }\DecValTok{25}

\CommentTok{# Create a variable `p` that contains 100 proportions ranging from 0 to 1 using the `seq` function}
\NormalTok{p <-}\StringTok{ }\KeywordTok{seq}\NormalTok{(}\DecValTok{0}\NormalTok{, }\DecValTok{1}\NormalTok{, }\DataTypeTok{length =} \DecValTok{100}\NormalTok{)}

\CommentTok{# Create a variable `se` that contains the standard error of each sample average}
\NormalTok{se <-}\StringTok{ }\KeywordTok{sqrt}\NormalTok{(p}\OperatorTok{*}\NormalTok{(}\DecValTok{1}\OperatorTok{-}\NormalTok{p)}\OperatorTok{/}\NormalTok{N)}

\CommentTok{# Plot `p` on the x-axis and `se` on the y-axis}
\KeywordTok{plot}\NormalTok{(p, se)}
\end{Highlighting}
\end{Shaded}

\includegraphics{01_-_Parameters_and_Estimates_files/figure-latex/unnamed-chunk-1-1.pdf}

\subsection{\texorpdfstring{\textbf{Multiple plots of se versus
p}}{Multiple plots of se versus p}}\label{multiple-plots-of-se-versus-p}

Using the same code as in the previous exercise, create a for-loop that
generates three plots of \texttt{p} versus \texttt{se} when the sample
sizes equal N=25, N=100, and N=1000.

Instructions

\begin{itemize}
\tightlist
\item
  Your for-loop should contain two lines of code to be repeated for
  three different values of N.
\item
  The first line within the for-loop should use the \texttt{sqrt}
  function to generate a vector of standard errors \texttt{se} for all
  values of \texttt{p}.
\item
  The second line within the for-loop should use the \texttt{plot}
  function to generate a plot with \texttt{p} on the x-axis and
  \texttt{se} on the y-axis.
\item
  Use the \texttt{ylim} argument to keep the y-axis limits constant
  across all three plots. The lower limit should be equal to 0 and the
  upper limit should equal the highest calculated standard error across
  all values of \texttt{p} and \texttt{N}.
\end{itemize}

\begin{Shaded}
\begin{Highlighting}[]
\CommentTok{# The vector `p` contains 100 proportions of Democrats ranging from 0 to 1 using the `seq` function}
\NormalTok{p <-}\StringTok{ }\KeywordTok{seq}\NormalTok{(}\DecValTok{0}\NormalTok{, }\DecValTok{1}\NormalTok{, }\DataTypeTok{length =} \DecValTok{100}\NormalTok{)}

\CommentTok{# The vector `sample_sizes` contains the three sample sizes}
\NormalTok{sample_sizes <-}\StringTok{ }\KeywordTok{c}\NormalTok{(}\DecValTok{25}\NormalTok{, }\DecValTok{100}\NormalTok{, }\DecValTok{1000}\NormalTok{)}

\CommentTok{# Write a for-loop that calculates the standard error `se` for every value of `p` for each of the three samples sizes `N` in the vector `sample_sizes`. Plot the three graphs, using the `ylim` argument to standardize the y-axis across all three plots.}
\ControlFlowTok{for}\NormalTok{(N }\ControlFlowTok{in}\NormalTok{ sample_sizes)\{}
\NormalTok{  se <-}\StringTok{ }\KeywordTok{sqrt}\NormalTok{(p}\OperatorTok{*}\NormalTok{(}\DecValTok{1}\OperatorTok{-}\NormalTok{p)}\OperatorTok{/}\NormalTok{N)}
  \KeywordTok{plot}\NormalTok{(p, se, }\DataTypeTok{ylim =} \KeywordTok{c}\NormalTok{(}\DecValTok{0}\NormalTok{,}\FloatTok{0.5}\OperatorTok{/}\KeywordTok{sqrt}\NormalTok{(}\DecValTok{25}\NormalTok{)))}
\NormalTok{\}}
\end{Highlighting}
\end{Shaded}

\includegraphics{01_-_Parameters_and_Estimates_files/figure-latex/unnamed-chunk-2-1.pdf}
\includegraphics{01_-_Parameters_and_Estimates_files/figure-latex/unnamed-chunk-2-2.pdf}
\includegraphics{01_-_Parameters_and_Estimates_files/figure-latex/unnamed-chunk-2-3.pdf}

\subsection{\texorpdfstring{\textbf{Expected value of
d}}{Expected value of d}}\label{expected-value-of-d}

Our estimate for the difference in proportions of Democrats and
Republicans is \(\ d=\bar{X}−(1−\bar{X})\).

Which derivation correctly uses the rules we learned about sums of
random variables and scaled random variables to derive the expected
value of d?

Possible Answers

\begin{itemize}
\tightlist
\item
  \(\ E[\bar{X}−(1−\bar{X})]=E[2\bar{X}−1] =2E[\bar{X}]−1 = N(2p−1) = Np−N(1−p)\)
\item
  \(\ E[\bar{X}−(1−\bar{X})]=E[\bar{X}−1] =E[\bar{X}]−1 =p−1\)
\item
  \(\ E[\bar{X}−(1−\bar{X})]=E[2\bar{X}−1] =2E[\bar{X}]−1 =2\sqrt{p(1−p)}−1 =p−(1−p)\)
\item
  \(\ E[\bar{X}−(1−\bar{X})]=E[2\bar{X}−1] =2E[\bar{X}]−1 =2p−1 =p−(1−p)\)
  {[}X{]}
\end{itemize}

\subsection{\texorpdfstring{\textbf{Standard error of
d}}{Standard error of d}}\label{standard-error-of-d}

Our estimate for the difference in proportions of Democrats and
Republicans is \(\ d=\bar{X}−(1−\bar{X})\).

Which derivation correctly uses the rules we learned about sums of
random variables and scaled random variables to derive the standard
error of \(\ d\)?

Possible Answers

\begin{itemize}
\tightlist
\item
  \(\ SE[\bar{X}−(1−\bar{X})]=SE[2\bar{X}−1] =2SE[\bar{X}] =2\sqrt{p/N}\)
\item
  \(\ SE[\bar{X}−(1−\bar{X})]=SE[2\bar{X}−1]=2SE[\bar{X}−1]=2\sqrt{p(1−p)/N}−1\)
\item
  \(\ SE[\bar{X}−(1−\bar{X})]=SE[2\bar{X}−1] =2SE[\bar{X}] =2\sqrt{p(1−p)/N}\)
  {[}X{]}
\item
  \(\ SE[\bar{X}−(1−\bar{X})]=SE[\bar{X}−1] =SE[\bar{X}] =\sqrt{p(1−p)/N}\)
\end{itemize}

\subsection{\texorpdfstring{\textbf{Standard error of the
spread}}{Standard error of the spread}}\label{standard-error-of-the-spread}

Say the actual proportion of Democratic voters is \(\ p=0.45\). In this
case, the Republican party is winning by a relatively large margin of
\(\ d=−0.1\), or a 10\% margin of victory. What is the standard error of
the spread \(\ 2\bar{X}−1\) in this case?

Use the \texttt{sqrt} function to calculate the standard error of the
spread \(\ 2\bar{X}−1\).

\begin{Shaded}
\begin{Highlighting}[]
\CommentTok{# `N` represents the number of people polled}
\NormalTok{N <-}\StringTok{ }\DecValTok{25}

\CommentTok{# `p` represents the proportion of Democratic voters}
\NormalTok{p <-}\StringTok{ }\FloatTok{0.45}

\CommentTok{# Calculate the standard error of the spread. Print this value to the console.}
\KeywordTok{print}\NormalTok{(}\DecValTok{2}\OperatorTok{*}\KeywordTok{sqrt}\NormalTok{(p}\OperatorTok{*}\NormalTok{(}\DecValTok{1}\OperatorTok{-}\NormalTok{p)}\OperatorTok{/}\NormalTok{N))}
\end{Highlighting}
\end{Shaded}

\begin{verbatim}
## [1] 0.1989975
\end{verbatim}

\subsection{\texorpdfstring{\textbf{Sample
size}}{Sample size}}\label{sample-size}

So far we have said that the difference between the proportion of
Democratic voters and Republican voters is about 10\% and that the
standard error of this spread is about 0.2 when N=25. Select the
statement that explains why this sample size is sufficient or not.

Possible Answers

\begin{itemize}
\tightlist
\item
  This sample size is sufficient because the expected value of our
  estimate \(\ 2\bar{X}−1\) is d so our prediction will be right on.
\item
  This sample size is too small because the standard error is larger
  than the spread. {[}X{]}
\item
  This sample size is sufficient because the standard error of about 0.2
  is much smaller than the spread of 10\%.
\item
  Without knowing \texttt{p}, we have no way of knowing that increasing
  our sample size would actually improve our standard error.
\end{itemize}


\end{document}
